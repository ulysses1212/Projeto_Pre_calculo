	\section{Produtos notáveis}
	
	Produtos notáveis são multiplicações em que os fatores (que estão sendo multiplicados) são polinômios. Existem cinco produtos notáveis mais relevantes e que serão utilizados com mais frequência: quadrado da soma de dois termos, quadrado da diferença de dois termos, produto da soma pela diferença de dois termos, cubo da soma de dois termos e cubo da diferença de dois termos. Cada um deles serão vistos em detalhe.\\
	\subsection{Principais produtos notáveis}

    \noindent
	\textbf{Quadrado da soma de dois termos}
	
	Para obter a expressão que representa o quadrado da soma de dois termos, basta representar de forma algébrica a frase que nomeia o produto notável.    

        \begin{tcolorbox}[colback=white,colframe=minha_cor,coltitle=black,title=Quadrado da soma de dois termos] 
        \centering
        \text{O quadrado:} $( \;\;\;)^2$ \\[0.25cm]
        \text{Da soma:} $(\;\;\;+\;\;\;)^2$ \\[0.25cm]
        \text{De dois termos:} $( a + b )^2$ 
        \end{tcolorbox}
	
	\[
	( a + b )^2 = a^2 + 2ab + b^2
	\]
	
	\textit{“O quadrado da soma de dois termos é igual ao quadrado do primeiro termo, mais duas vezes o produto do primeiro pelo segundo termo, mais o quadrado do segundo termo.''}
	
	Desenvolvendo algebricamente a expressão $( a + b )^2$, tem-se:
	
	\[
	( a + b )^2 = ( a + b )( a + b ) = a \cdot a + a \cdot b + b \cdot a + b \cdot b = a^2 + a \cdot b + a \cdot b + b^2 = a^2 + 2ab + b^2
	\]


        \begin{texample}
        \centering
        \tcbhighmath{( x + 3y )^2 = x^2 + 2 \cdot x \cdot 3y +(3y)^2 = x^2 + 6xy + 9y^2}
        \tcbhighmath{( 7x + 1 )^2 = (7x)^2 + 2 \cdot 7x \cdot 1 +1^2 = 49x^2 + 14x + 1}
        \tcbhighmath{( a^2 + 2bc )^2 = (a^2)^2 + 2 \cdot a^2 \cdot 2bc +(2bc)^2 = a^4 + 4a^2bc + 4b^2c^2}
        \tcbhighmath{\left( 2m + \dfrac{3}{4} \right)^2 = (2m)^2 + 2 \cdot 2m \cdot \left(\dfrac{3}{4}\right) +\left(\dfrac{3}{4}\right)^2 = 4m^2 + 3m + \left(\dfrac{9}{16}\right) }
        \end{texample}

    \noindent
	\textbf{Quadrado da diferença de dois termos}
 
	Para obter a expressão que representa o quadrado da diferença de dois termos, deve-se transcrever esse produto em linguagem algébrica.

        \begin{tcolorbox}[colback=white,colframe=minha_cor,coltitle=black,title=O quadrado da diferença de dois termos] 
        \centering
        \text{O quadrado:} $( \;\;\;)^2$ \\[0.25cm]
        \text{Da diferença:} $(\;\;\;-\;\;\;)^2$ \\[0.25cm]
        \text{De dois termos:} $( a - b )^2$ 
        \end{tcolorbox}
	
	\[
	( a - b )^2 = a^2 - 2ab + b^2
	\]
	
	\textit{“O quadrado da diferença de dois termos é igual ao quadrado do primeiro termo, \textbf{menos} duas vezes o produto do primeiro termo pelo segundo termo mais o quadrado do segundo termo.''}
	
	Desenvolvendo algebricamente essa expressão, tem-se:
 
	\[
	( a - b )^2 = ( a - b )( a - b ) = a \cdot a - a \cdot b - b \cdot a + b \cdot b = a^2 - a \cdot b - a \cdot b + b^2 = a^2 - 2ab + b^2
	\]

        \begin{texample}
        \centering
        \tcbhighmath{( 7x - 4 )^2 = (7x)^2 - 2 \cdot 7x \cdot 4 +4^2 = 49x^2 - 56x + 16}
        \tcbhighmath{( x^3 - xy )^2 = (x^3)^2 - 2 \cdot x^3 \cdot xy + (xy)^2 = x^6 - 2x^4y + x^2y^2}
        \tcbhighmath{( a^2 - 2bc )^2 = (a^2)^2 - 2 \cdot a^2 \cdot 2bc +(2bc)^2 = a^4 - 4a^2bc + 4b^2c^2}
        \tcbhighmath{\left(\dfrac{1}{5}p - 2q\right)^2 = \left(\dfrac{1}{5}p \right)^2 - 2 \cdot \dfrac{1}{5}p \cdot 2q + (2q)^2 = \dfrac{1}{25}p^2 - \left(\dfrac{4}{5}pq\right) + 4q^2 }
        \end{texample} 

    \noindent
	\textbf{Produto da Soma pela Diferença de Dois Termos}
	
	Em termos algébrico, tem-se:
	\begin{tcolorbox}[colback=white,colframe=minha_cor,coltitle=black,title=Produto da Soma pela Diferença de Dois Termos] 
        \centering
        \text{O produto:} $( \;\;\;) \times ( \;\;\;) $ \\[0.25cm]
        \text{Da soma:} $(\;\;\;+\;\;\;) \times ( \;\;\;)$ \\[0.25cm]
        \text{Pela diferença:} $(\;\;\;+\;\;\;) \times (\;\;\;-\;\;\;)$ \\[0.25cm]
        \text{De dois termos:} $( a + b ) \times ( a - b )$
        \end{tcolorbox}
	
	\[
	( a + b ) ( a - b ) = a^2 - b^2
	\]
	
	\textit{“O produto da soma pela diferença de dois termos é igual ao quadrado do primeiro termo menos o quadrado do segundo termo''}
	
	Desenvolvendo algebricamente, tem-se:
	
	\[
	( a + b ) \times ( a - b ) = a \cdot a - a \cdot b + b \cdot a - b \cdot b = a^2 - b^2
	\]

        \begin{texample}
        \centering
        \tcbhighmath{( 3a + x ) \times ( 3a - x ) = (3a)^2 - x^2 = 9a^2 - x^2}
        \tcbhighmath{( x^2 + 5p ) \times ( x^2 - 5p ) = (x^2)^2 - (5p)^2 = x^4 - 25p^2}
        \tcbhighmath{( 10 + ab^3 ) \times ( 10 - ab^3 ) = 10^2 - (ab^3)^2 = 100 - a^2b^6}
        \tcbhighmath{\left( b^3 + \dfrac{3}{5}c \right) \times \left( b^3 - \dfrac{3}{5}c \right) = (b^3)^2 - \left(\dfrac{3}{5}c \right)^2 = b^6 - \dfrac{9}{25}c^2 }
        \end{texample}

    \noindent
	\textbf{O Cubo da Soma de Dois Termos}
	
	A notação algébrica desse produto notável é:

        \begin{tcolorbox}[colback=white,colframe=minha_cor,coltitle=black,title=O Cubo da Soma de Dois Termos] 
        \centering
        \text{O cubo:} $( \;\;\;)^3$ \\[0.25cm]
        \text{Da diferença:} $(\;\;\;+\;\;\;)^3$ \\[0.25cm]
        \text{De dois termos:} $( a + b )^3$
        \end{tcolorbox}
	
	\[
	( a + b )^3 = a^3 + 3 a^2 b + 3 a b^2 + b^3 
	\]
	
	\textit{“O cubo da soma de dois termos é igual ao cubo do primeiro termo mais três vezes o produto do primeiro termo ao quadrado pelo segundo termo mais três vezes o produto do primeiro termos pelo quadrado do segundo termo mais o cubo do segundo termo''}
	
	O desenvolvimento algébrico será omitido neste item.

        \begin{texample}
        \centering
        \tcbhighmath{( 2x + 1 )^3 = (2x)^3 + 3(2x)^2 \cdot 1 + 3 \cdot 2x(1)^2 + (1)^3  = 8x^3 + 12x^2 + 6x + 1 }
        \tcbhighmath{\left( \dfrac{1}{2} + \dfrac{2}{x} \right)^3 = \left( \dfrac{1}{2} \right)^3 + 3 \cdot \left( \dfrac{1}{2} \right)^2 \cdot \dfrac{2}{x}  + 3 \cdot \dfrac{1}{2} \cdot \left(\dfrac{2}{x} \right)^2 + \left(\dfrac{2}{x} \right)^3 = \dfrac{1}{8} + \dfrac{6}{4x} + \dfrac{12}{2x^2} + \dfrac{8}{x^3}}
        \end{texample}

    \noindent
	\textbf{O Cubo da Diferença de Dois Termos}
	
	A notação algébrica desse produto notável é:

        \begin{tcolorbox}[colback=white,colframe=minha_cor,coltitle=black,title=O Cubo da Diferença de Dois Termos] 
        \centering
        \text{O cubo:} $( \;\;\;)^3$ \\[0.25cm]
        \text{Da diferença:} $(\;\;\;-\;\;\;)^3$ \\[0.25cm]
        \text{De dois termos:} $( a - b )^3$
        \end{tcolorbox}
	
	\[
	( a - b )^3 = a^3 - 3 a^2 b + 3 a b^2 - b^3 
	\]
	
	\textit{"O cubo da diferença de dois termos é igual ao cubo do primeiro termo \textbf{menos} três vezes o produto do primeiro termo ao quadrado pelo segundo termo \textbf{mais} três vezes o produto do primeiro termos pelo quadrado do segundo termo \textbf{menos} o cubo do segundo termo''}
	
	O desenvolvimento algébrico será omitido neste item.

        \begin{texample}
        \centering
        \tcbhighmath{( 1 - 2z )^3 = (1)^3 - 3(1)^2 \times 2z + 3 \times 1(2z)^2 - (2z)^3  = 1 - 6z + 12z^2 - 8z^3 }
        \tcbhighmath{\left( 2 - \dfrac{x}{3} \right)^3 = (2)^3 - 3(2)^2 \times \dfrac{x}{3}  + 3 \times 2\left(\dfrac{x}{3} \right)^2 - \left(\dfrac{x}{3} \right)^3 = 8 - 4x + \dfrac{2x^2}{3} - \dfrac{x^3}{27}}
        \end{texample}
        
	\section{Fatoração}

    \noindent
	\textbf{Técnicas de fatoração}
	
	Fatoração é um processo utilizado na matemática que consiste em representar um número ou uma expressão como produto de fatores. Ao escrever um polinômio como a multiplicação de outros polinômios, frequentemente é possível simplificar a expressão. As principais técnicas de fatoração serão exemplificadas a seguir.\\

    \noindent
	\textbf{Fator comum em evidência}
	
	Esse tipo de fatoração é usado quando existe um fator que se repete em todos os termos do polinômio. Esse fator, que pode conter número e letras, será colocado na frente dos parênteses. Dentro dos parênteses ficará o resultado da divisão de cada termo do polinômio pelo fator comum. Isto é,
	
	\[
	ax + bx = x(a+b)
	\]

        \begin{texample}
        \centering
        \tcbhighmath{y^5 - 2y^4 + y^3 = y^3(y^2 -2y + 1)}
        \tcbhighmath{x(m + n) - y(m + n) = (m + n)(x - y)}
        \tcbhighmath{10ax + 15bx = x(10a + 15b) = 5x(2a + 3b)}
        \tcbhighmath{8a^4b^2 - 20a^3b^5 = a^3(8ab^2 - 20b^5) = a^3b^2(8a -20b^3) = 4a^3b^2(2a - 5b^3)}
        \end{texample}

    \noindent
	\textbf{Agrupamento}
	
	No polinômio que não exista um fator que se repita em todos os termos, é possível usar a fatoração por agrupamento. Para isso, deve-se identificar os termos que podem ser agrupados por fatores comuns. Nesse tipo de fatoração, os fatores comuns dos agrupamentos são colocados em evidência. Isto é,
	
	\[
	ax + bx + ay + by = (a + b)(x +y)
	\]
 
	Primeiramente, deve-se agrupar os temos que tem um fator comum:
	\[
	ax + bx + ay + by = (ax + bx) + (ay + by)
	\]
 
	Depois o fator comum é colocado em evidência em cada agrupamento:
	\[
	ax + bx + ay + by = (ax + bx) + (ay + by) = x(a + b) + y(a + b)
	\]
 
	E por fim, novamente, o fator comum é colocado em evidência:
	\[
	ax + bx + ay + by = (ax + bx) + (ay + by) = x(a + b) + y(a + b) = (a + b)(x +y)
	\]
	
	Alguns exemplos são apresentados a seguir.

        \begin{texample}
        \centering
         \tcbhighmath{y^3 + y^2 + y +1 = y^2 (y + 1 ) + (y +1)= (y +1)(y^2 +1)}
        \tcbhighmath{ax - 2x + 5a - 10 = x(a-2) + 5(a-2) = (a - 2)(x + 5)}
        \tcbhighmath{2bc + 5c^2 -10b -25c = 2b(c - 5) + 5c(c - 5) = (c - 5)(2b + 5c)}
        \end{texample}

    \noindent
	\textbf{Trinômio Quadrado Perfeito}
	
	Trinômios são polinômios com três termos. Note que os trinômios quadrados perfeitos são os resultados dos produtos notáveis apresentados anteriormente
	\[
	x^2 + 2xy + y^2 = (x + y)^2
	\]
	\[
	x^2 - 2xy + y^2 = (x - y)^2
	\]
	É necessário reconhecer se um trinômio é ou não um quadrado perfeito. Seja o trinômio:
	\[
	a^2 +10ab + 25b^2
	\]
	A raiz dos termos ao quadrado é extraída, ou seja,
	\[
	\sqrt{a^2} = a \;\;\; e \;\;\; \sqrt{25 \cdot b^2} = 5b
	\]
	esses dois resultados são multiplicados por dois,
	\[
	2 \cdot a \cdot 5b = 10ab,
	\]
	que é exatamente o termo do meio do trinômio. Portanto, ele é um quadrado perfeito e a sua fatoração é dada por
	\[
	(a + 5b)^2
	\]

        \begin{texample}
        \centering
        \tcbhighmath{y^2 - 14y + 49 = (y - 7)^2}
        \tcbhighmath{25p^2 + 30px +9x^2 = (5p + 3x)^2}
        \tcbhighmath{a^6 + 22a^3 + 121 = (a^3 + 11)^2}
        \end{texample}

    \noindent
	\textbf{Diferença de Dois Quadrados}
	
	Para fatorar polinômios do tipo $a^2 - b^2$ o produto notável da soma pela diferença é usado. Assim, a fatoração de polinômios desse tipo será:
	
	\[
	x^2 - y^2 = (x + y)\cdot(x - y)
	\]
	
	Para fatorar, calcula-se a raiz quadrada dos dois termos. Depois, escrever o produto da soma dos valores encontrados pela diferença desses valores.
	
	Alguns exemplos são apresentados a seguir.

        \begin{texample}
        \centering
        \tcbhighmath{x^2 - 25 = x^2 - 5^2 = (x + 5)(x - 5)}
        \tcbhighmath{\left(\frac{1}{4}a^2 - x^2y^2\right) = \left( \frac{1}{2}a^2\right) - (xy)^2 - (xy)^2 = \left(\frac{1}{2}a + xy\right) \left(\frac{1}{2}a - xy\right)}
        \tcbhighmath{(a + 7)^2 - 36 = (a + 7)^2 - 6^2 = [(a +7) + 6][(a +7) - 6] = (a + 13)(a + 1)}
        \end{texample}
        
	\section{Exercícios}

    \noindent
	\textbf{Produtos Notáveis}
	
	\begin{enumerate}
		\item Utilizando as regras dos produtos notáveis, calcule:\\
		\begin{tasks}(2)
			\task $( x + 3 )^2 $ \\[-0.25cm]
			\task $( a + b )^2 $ \\[-0.25cm]
			\task $( 5y - 1 )^2 $ \\[-0.25cm]
			\task $( x^2 - 6 )^2 $ \\[-0.25cm]
			\task $( 2x + 7 )^2 $ \\[-0.25cm]
			\task $( 9x + 1 )(9x - 1) $ \\[-0.25cm]
			\task $( a^2 -xy )^2 $ \\[-0.25cm]
			\task $( 3x - \frac{1}{6}y )^2 $ \\[-0.25cm]
			\task $( 2x^2 + 3xy )^2 $ \\[-0.25cm]
			\task $ (x^3y - xy^3)^2 $ \\[-0.25cm]
			\task $ (3y - 5)^2 $ \\[-0.25cm]
			\task $ (5 + 8b)^2 $ \\[-0.25cm]
			\task $ (a^4x^2 + a^2x^4) (a^4x^2 - a^2x^4)$ \\[-0.25cm]
			\task $\left( 2x^2 - \dfrac{3}{5} \right) \left( 2x^2 + \dfrac{3}{5} \right) $ \\[-0.25cm]
			\task $( 2x^3 + 3y^2 ) ( 2x^3 - 3y^2 )$
		\end{tasks}
	
	\hspace{-1.2cm} \textbf{Fatoração}
	
	\item Fatore as seguintes expressões:
	
	\begin{tasks}(2)
		\task $x^2 - 4$ \\[-0.25cm]
		\task $y^2 - 36$ \\[-0.25cm]
		\task $9x^2 - 16$ \\[-0.25cm]
		\task $81x^2 - 64$ \\[-0.25cm]
		\task $y^2 - 25x^2$ \\[-0.25cm]
		\task $4x^2 - 25a^2$ \\[-0.25cm]
		\task $4x^2 - 20x - 25$ \\[-0.25cm]
		\task $16y^2 - x^4$ \\[-0.25cm]
		\task $25m^2 + 20m + 4$ \\[-0.25cm]
		\task $25x^2 - \dfrac{10}{3}x + \dfrac{1}{9}$
	\end{tasks}
	
	\item Observe a fatoração seguinte:
	\[
	a^4 - 1 = (a^2 + 1)(a^2 - 1) = (a^2 + 1)(a + 1)(a - 1)
	\]
	Agora, decomponha num produto de três fatores.
	\begin{tasks}
		\task $x^4 - 1$
		\task $81a^4 - 1$
	\end{tasks}
	
	\item Efetue as divisões seguintes, fatorando o dividendo:
	\begin{tasks}
		\task $\dfrac{x^2-14x+49}{x-7}$
		\task $\dfrac{x^2 - 16}{x + 4}$
	\end{tasks}
\end{enumerate}
	
	\section{Respostas dos exercícios}
    \noindent
	\textbf{Produtos Notáveis}

    \begin{enumerate}
        \item \begin{tasks}(2)
		\task $ x^2 + 6x + 9 $ \\[-0.25cm]
		\task $ a^2 + 2ab + b^2 $ \\[-0.25cm]
		\task $ 25y^2 - 10y + 1$ \\[-0.25cm]
		\task $ x^4 - 12x^2 + 36 $ \\[-0.25cm]
		\task $ 4x^2 + 28x + 49 $ \\[-0.25cm]
		\task $ 81x^2 - 1 $ \\[-0.25cm]
		\task $ a^4 - 2a^2xy + x^2y^2 $ \\[-0.25cm]
		\task $ 9x^2 - xy + \frac{y^2}{36} $ \\[-0.25cm]
		\task $ 4x^4 +12x^3y + 9x^2y^2 $ \\[-0.25cm]
		\task $ x^6y^2 - 2x^4y^4 + x^2y^6 $ \\[-0.25cm]
		\task $ 9y^2 - 30y + 25 $ \\[-0.25cm]
		\task $ 25 + 80b + 64b^2 $ \\[-0.25cm]
		\task $ a^8x^4 - a^4x^8 $ \\[-0.25cm]
		\task $ 4x^4 - \frac{9}{25} $ \\[-0.25cm]
		\task $ 4x^6 - 9y^4 $ \\
	\end{tasks}
    \noindent
	\hspace{-1.1cm} \textbf{Fatoração}
		\item 
        \begin{tasks}(2)
			\task $ (x+2)(x-2)$ \\[-0.25cm]
			\task $ (y+6)(y-6)$ \\[-0.25cm]
			\task $ (3x+4)(3x-4)$ \\[-0.25cm]
			\task $ (9x+8)(9x-8)$ \\[-0.25cm]
			\task $ (y+5x)(y-5x)$ \\[-0.25cm]
			\task $ (2x+5a)(2x-5a)$ \\[-0.25cm]
			\task $ (x+4)^2$ \\[-0.25cm]
			\task $ (4y + x^2)(4y - x^2)$ \\[-0.25cm]
			\task $ (5m + 2)^2$ \\[-0.25cm]
			\task $ (5x - \frac{1}{3})^2$
		\end{tasks}
  
		\item
		\begin{tasks}
			\task $ (x^2 + 1)(x + 1)(x - 1)$ \\[-0.25cm]
			\task $ (9a^2 + 1)(3a + 1)(3a - 1)$
		\end{tasks}
  
		\item
		\begin{tasks}
			\task $ (x - 7)$ \\[-0.25cm]
			\task $ (x - 4)$
		\end{tasks}
\end{enumerate}